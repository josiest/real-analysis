\documentclass{article}
\usepackage[utf8]{inputenc}
\usepackage{geometry}
\geometry{margin=1in}

\usepackage{amssymb}
\usepackage{amsmath}
\usepackage{amsthm}

\usepackage{enumerate}

\begin{document}
\newcommand{\R}{\mathbb{R}}
\newcommand{\claim}{\par\noindent\textit{Claim:}\space}

\setcounter{section}{2}
\section{Numerical Sequences and Series}

\begin{enumerate}
\setcounter{enumi}{12}
\item \claim The Cuachy product of two absolutely convergent series is also
      absolutely convergent.

\begin{proof}
Let $\sum a_n, \sum b_n$ be two absolutely convergent series. Put
$A = \sum |a_n|$ and $B = \sum |b_n|$. Then their cauchy product
$\sum_{n=0}^{\infty} \sum_{k=0}^n |a_n b_n|$ converges to $AB$. Then by the
triangle inequality
\begin{equation*}
\sum_{n=0}^{\infty} \sum_{k=0}^n | a_n b_n | \geq
\sum_{n=0}^{\infty} \left| \sum_{k=0}^n a_n b_n \right|
\end{equation*}

Which means that the Cauchy product of two convergent series must converge by
comparison.
\end{proof}

\setcounter{enumi}{15}
\item Fix $\alpha > 0$ and choose $x_1 > \sqrt{\alpha}$. Define
\begin{equation*}
x_{n+1} = \frac{1}{2} \left(x_n + \frac{\alpha}{x_n}\right)
\end{equation*}
\begin{enumerate}[a.]
\item Prove that $x_n$ is monotonically decreasing and that
      $\lim x_n = \sqrt{\alpha}$.

\claim $x_n^2 > \alpha$
\begin{proof}
This is true for $x_1$ by its definition, so let $n \geq 2$ and suppose
$x_n > \sqrt{\alpha}$.

\begin{equation*}
\begin{split}
(x_n - \sqrt{\alpha})^2 &> 0 \\
x_n^2 + \alpha &> 2\sqrt{\alpha} x_n \\
\frac{1}{2}\left(x_n + \frac{\alpha}{x_n}\right) &> \sqrt{\alpha} \\
x_{n+1} &> \sqrt{\alpha}
\end{split}
\end{equation*}
\end{proof}

\claim $x_n$ is monotonically decreasing.
\begin{proof}
Let $n \geq 1$.
\begin{equation*}
\begin{split}
 x_n^2 &> \alpha \\
2x_n^2 &> x_n^2 + \alpha \\
   x_n &> \frac{1}{2} \left(x_n + \frac{\alpha}{x_n}\right) \\
   x_n &> x_{n+1}
\end{split}
\end{equation*}
\end{proof}

\claim $x_n \to \sqrt{\alpha}$
\begin{proof}
$x_n$ is monotonically decreasing and bounded below by $\sqrt{\alpha}$, so it
must converge to some $x \in \R \geq \sqrt{\alpha}$. Suppose for contradiction
that $x > \sqrt{\alpha}$, and put $\varepsilon = \frac{\sqrt{x^2 - \alpha}}{2}$.
There must be some $N$ where if $n\geq N$ then $x < x_n < x+\varepsilon$. So
then we have

\begin{equation*}
\begin{split}
\varepsilon &< \sqrt{x^2 - \alpha} \\
\varepsilon^2 + \alpha &< x^2 \\
\varepsilon(x + \varepsilon) + \alpha &< x(x+\varepsilon) \\
x + \varepsilon + \frac{\alpha}{x+\varepsilon} &< 2x \\
\frac{1}{2}\left(x + \varepsilon + \frac{\alpha}{x+\varepsilon}\right) &< x
\end{split}
\end{equation*}
And since $x_n < x + \varepsilon$ it must also be the case that
\begin{equation*}
x_{n+1} = \frac{1}{2}\left(x_n + \frac{\alpha}{x_n}\right)
        < \frac{1}{2}\left(x+\varepsilon + \frac{\alpha}{x+\varepsilon}\right)
        < x
\end{equation*}
This contradicts that $x = \inf\{x_n\}$, so it must be the case that
$x = \sqrt{\alpha}$ exactly.
\end{proof}

\item Put $\varepsilon_n = x_n - \sqrt{\alpha}$
\claim \begin{equation*}
\varepsilon_{n+1} = \frac{\varepsilon_n^2}{2x_n}
                  < \frac{\varepsilon_n^2}{2\sqrt{\alpha}}
\end{equation*}
\begin{proof}
\begin{equation*}
\begin{split}
\varepsilon_{n+1} &= x_{n+1} - \sqrt{\alpha} \\
                  &= \frac{1}{2}\left(x_n + \frac{\alpha}{x_n}\right)
                   - \sqrt{\alpha} \\
                  &= \frac{x_n^2 - 2\sqrt{\alpha}x_n + \alpha}{2x_n} \\
                  &= \frac{(x_n-\sqrt{\alpha})^2}{2x_n}
                   = \frac{\varepsilon_n^2}{2x_n}
\end{split}
\end{equation*}
We showed earlier that $x_n$ is bounded below by $\sqrt{\alpha}$, so it follows
that
\begin{equation*}
\frac{\varepsilon_n^2}{2x_n} < \frac{\varepsilon_n^2}{2\sqrt{\alpha}}
\end{equation*}
A basic inductive argument shows that
\begin{equation*}
\varepsilon_{n+1} < 2\sqrt{\alpha}\left(
                        \frac{\varepsilon_1}{2\sqrt{\alpha}}
                    \right)^{2^n}
\end{equation*}
\end{proof}
\end{enumerate}
\end{enumerate}
\end{document}
